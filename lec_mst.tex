\PassOptionsToPackage{unicode}{hyperref}
\documentclass[aspectratio=1610, 11pt]{beamer}

\usepackage{amsmath}
\usepackage{amssymb}
\usetheme{tudo}

\title{Datenstrukturen, Algorithmen und Prorammierung~2}
\author[A.~Coja-Oghlan]{Amin Coja-Oghlan}
\institute[DAP2]{Lehrstuhl Informatik 2\\Fakult\"at f\"ur Informatik}

\newcommand\dist{\mathrm{dist}}
\renewcommand{\vec}[1]{\boldsymbol{#1}}
\newcommand\NULL{{\tt NULL}}
\newcommand\dd{\mathrm d}
\newcommand\eul{\mathrm e}
\newcommand\cA{\mathcal A}
\newcommand\cB{\mathcal B}
\newcommand\cC{\mathcal C}
\newcommand\cD{\mathcal D}
\newcommand\cE{\mathcal E}
\newcommand\cF{\mathcal F}
\newcommand\cG{\mathcal G}
\newcommand\cH{\mathcal H}
\newcommand\cI{\mathcal I}
\newcommand\cJ{\mathcal J}
\newcommand\cK{\mathcal K}
\newcommand\cL{\mathcal L}
\newcommand\cM{\mathcal M}
\newcommand\cN{\mathcal N}
\newcommand\cO{\mathcal O}
\newcommand\cP{\mathcal P}
\newcommand\cQ{\mathcal Q}
\newcommand\cR{\mathcal R}
\newcommand\cS{\mathcal S}
\newcommand\cT{\mathcal T}
\newcommand\cU{\mathcal U}
\newcommand\cV{\mathcal V}
\newcommand\cW{\mathcal W}
\newcommand\cX{\mathcal X}
\newcommand\cY{\mathcal Y}
\newcommand\cZ{\mathcal Z}
\newcommand\fA{\mathfrak A}
\newcommand\fB{\mathfrak B}
\newcommand\fC{\mathfrak C}
\newcommand\fD{\mathfrak D}
\newcommand\fE{\mathfrak E}
\newcommand\fF{\mathfrak F}
\newcommand\fG{\mathfrak G}
\newcommand\fH{\mathfrak H}
\newcommand\fI{\mathfrak I}
\newcommand\fJ{\mathfrak J}
\newcommand\fK{\mathfrak K}
\newcommand\fL{\mathfrak L}
\newcommand\fM{\mathfrak M}
\newcommand\fN{\mathfrak N}
\newcommand\fO{\mathfrak O}
\newcommand\fP{\mathfrak P}
\newcommand\fQ{\mathfrak Q}
\newcommand\fR{\mathfrak R}
\newcommand\fS{\mathfrak S}
\newcommand\fT{\mathfrak T}
\newcommand\fU{\mathfrak U}
\newcommand\fV{\mathfrak V}
\newcommand\fW{\mathfrak W}
\newcommand\fX{\mathfrak X}
\newcommand\fY{\mathfrak Y}
\newcommand\fZ{\mathfrak Z}
\newcommand\fa{\mathfrak a}
\newcommand\fb{\mathfrak b}
\newcommand\fc{\mathfrak c}
\newcommand\fd{\mathfrak d}
\newcommand\fe{\mathfrak e}
\newcommand\ff{\mathfrak f}
\newcommand\fg{\mathfrak g}
\newcommand\fh{\mathfrak h}
%\newcommand\fi{\mathfrak i}
\newcommand\fj{\mathfrak j}
\newcommand\fk{\mathfrak k}
\newcommand\fl{\mathfrak l}
\newcommand\fm{\mathfrak m}
\newcommand\fn{\mathfrak n}
\newcommand\fo{\mathfrak o}
\newcommand\fp{\mathfrak p}
\newcommand\fq{\mathfrak q}
\newcommand\fr{\mathfrak r}
\newcommand\fs{\mathfrak s}
\newcommand\ft{\mathfrak t}
\newcommand\fu{\mathfrak u}
\newcommand\fv{\mathfrak v}
\newcommand\fw{\mathfrak w}
\newcommand\fx{\mathfrak x}
\newcommand\fy{\mathfrak y}
\newcommand\fz{\mathfrak z}
\newcommand\vA{\vec A}
\newcommand\vB{\vec B}
\newcommand\vC{\vec C}
\newcommand\vD{\vec D}
\newcommand\vE{\vec E}
\newcommand\vF{\vec F}
\newcommand\vG{\vec G}
\newcommand\vH{\vec H}
\newcommand\vI{\vec I}
\newcommand\vJ{\vec J}
\newcommand\vK{\vec K}
\newcommand\vL{\vec L}
\newcommand\vM{\vec M}
\newcommand\vN{\vec N}
\newcommand\vO{\vec O}
\newcommand\vP{\vec P}
\newcommand\vQ{\vec Q}
\newcommand\vR{\vec R}
\newcommand\vS{\vec S}
\newcommand\vT{\vec T}
\newcommand\vU{\vec U}
\newcommand\vV{\vec V}
\newcommand\vW{\vec W}
\newcommand\vX{\vec X}
\newcommand\vY{\vec Y}
\newcommand\vZ{\vec Z}
\newcommand\va{\vec a}
\newcommand\vb{\vec b}
\newcommand\vc{\vec c}
\newcommand\vd{\vec d}
\newcommand\ve{\vec e}
\newcommand\vf{\vec f}
\newcommand\vg{\vec g}
\newcommand\vh{\vec h}
\newcommand\vi{\vec i}
\newcommand\vj{\vec j}
\newcommand\vk{\vec k}
\newcommand\vl{\vec l}
\newcommand\vm{\vec m}
\newcommand\vn{\vec n}
\newcommand\vo{\vec o}
\newcommand\vp{\vec p}
\newcommand\vq{\vec q}
\newcommand\vr{\vec r}
\newcommand\vs{\vec s}
\newcommand\vt{\vec t}
\newcommand\vu{\vec u}
\renewcommand\vv{\vec v}
\newcommand\vw{\vec w}
\newcommand\vx{\vec x}
\newcommand\vy{\vec y}
\newcommand\vz{\vec z}
\renewcommand\AA{\mathbb A}
\newcommand\NN{\mathbb N}
\newcommand\ZZ{\mathbb Z}
\newcommand\PP{\mathbb P}
\newcommand\QQ{\mathbb Q}
\newcommand\RR{\mathbb R}
\newcommand\RRpos{\mathbb R_{\geq0}}
\newcommand\QQpos{\mathbb Q_{\geq0}}
\renewcommand\SS{\mathbb S}
\newcommand\CC{\mathbb C}
\newcommand{\ord}{\mathrm{ord}}
\newcommand{\id}{\mathrm{id}}
\newcommand{\pr}{\mathrm{P}}
\newcommand{\Vol}{\mathrm{vol}}
\newcommand\norm[1]{\left\|{#1}\right\|} 
\newcommand\sign{\mathrm{sign}}
\newcommand{\eps}{\varepsilon}
\newcommand{\abs}[1]{\left|#1\right|}
\newcommand\bc[1]{\left({#1}\right)} 
\newcommand\cbc[1]{\left\{{#1}\right\}} 
\newcommand\bcfr[2]{\bc{\frac{#1}{#2}}} 
\newcommand{\bck}[1]{\left\langle{#1}\right\rangle} 
\newcommand\brk[1]{\left\lbrack{#1}\right\rbrack} 
\newcommand\scal[2]{\bck{{#1},{#2}}} 
\newcommand{\vecone}{\mathbb{1}}
\newcommand{\tensor}{\otimes}
\newcommand{\diag}{\mathrm{diag}}
\newcommand{\ggt}{\mathrm{ggT}}
\newcommand{\kgv}{\mathrm{kgV}}
\newcommand{\trans}{\top}
\newcommand{\Karonski}{Karo\'nski}
\newcommand{\Erdos}{Erd\H{o}s}
\newcommand{\Renyi}{R\'enyi}
\newcommand{\Lovasz}{Lov\'asz}
\newcommand{\Juhasz}{Juh\'asz}
\newcommand{\Bollobas}{Bollob\'as}
\newcommand{\Furedi}{F\"uredi}
\newcommand{\Komlos}{Koml\'os}
\newcommand{\Luczak}{\L uczak}
\newcommand{\Kucera}{Ku\v{c}era}
\newcommand{\Szemeredi}{Szemer\'edi}

\newcommand{\mytitle}{Minimal spannende B\"aume}

\begin{document}

\frame[plain]{\titlepage}

\begin{frame}\frametitle{\mytitle}
	\begin{exampleblock}{Worum geht es?}
		\begin{itemize}
			\item jeder zusammenh\"angende Graph besitzt einen spannenden Baum
			\item in einem gewichteten Graphen sind wir interessiert an einem \alert{minimalen} spannenden Baum
			\item dazu lernen wir einen \emph{Greedy-Algorithmus} kennen
		\end{itemize}
	\end{exampleblock}
\end{frame}

\begin{frame}\frametitle{\mytitle}
	\begin{overprint}
		\onslide<1>
\begin{block}{Definition}
		\begin{itemize}
			\item ein \alert{gewichteter Graph} ist ein Graph $G=(V,E)$ zusammen mit einer Funktion $c:E\to \QQpos$.
			\item das \alert{Gewicht} einer Kantenmenge $F\subseteq E$ ist
				\begin{align*}
					\sum_{e\in F}c(e)
				\end{align*}
			\item ein \alert{minimal spannender Baum} von $G$ ist ein spannender Baum $T=(V,E_T)$ von $G$ mit
				\begin{align*}
					c(E_T)=\min\cbc{c(E_B):B=(V,E_B)\mbox{ ist spannender Baum von }G}
				\end{align*}
		\end{itemize}
	\end{block}
		\onslide<2>
\begin{exampleblock}{Bemerkung}
		\begin{itemize}
			\item f\"ur algorithmische Zwecke nehmen wir stets an, da\ss\ die Gewichtsfunktion $c$ rationale Werte annimmt
			\item diese rationalen Werte werden dann exakt eingegeben, d.h.\ als Paare ganzer Zahlen
				\begin{align*}
					\mbox{(Z\"ahler,Nenner)}
				\end{align*}
			\item \"aquivalent k\"onnen wir annehmen, da\ss\ alle Gewichte ganzzahlig sind
		\end{itemize}
	\end{exampleblock}
	\end{overprint}
\end{frame}

\begin{frame}\frametitle{\mytitle}
	\begin{overprint}
		\onslide<1>
		\begin{exampleblock}{Algorithmus Kruskal}
			\begin{enumerate}
				\item pr\"ufe, ob $G$ zusammenh\"angend ist; wenn nicht, gib eine Fehlermeldung aus
				\item sortiere die Kanten von $G$ aufsteigend nach Gewichten:
					\begin{align*}
						c(e_1)\leq c(e_2)\leq\cdots \leq c(e_m)
					\end{align*}
				\item initialisiere $T=(V,\emptyset)$
				\item f\"ur $i=1,\ldots,m$
				\item $\quad$ wenn $e_i$ verschiedenene Komponenten von $T$ verbindet
				\item $\quad\quad$ f\"uge $e_i$ zu $T$ hinzu
				\item gib $T$ aus
			\end{enumerate}
		\end{exampleblock}
		\onslide<2>
		\begin{exampleblock}{Bemerkungen}
			\begin{itemize}
				\item Schritt 1 kann mit Tiefensuche in Zeit $O(|E|+|V|)$ durchgef\"uhrt werden
				\item f\"ur Schritt 2 verwenden wir {\tt Heapsort}; Laufzeit $O(|E|\log|E|)$
				\item f\"ur Schritte 5 und 6 f\"uhren wir Buch \"uber die Komponenten von $T$
				\item dabei f\"uhren wir die Komponentengr\"o\ss en mit
				\item jedesmal, wenn eine Kante eingef\"ugt wird, wird die \emph{kleinere} Komponente zur gr\"o\ss eren hinzugef\"ugt
			\end{itemize}
		\end{exampleblock}
		\onslide<3>
		\begin{exampleblock}{Bemerkungen}
			\begin{itemize}
				\item der Kruskal-Algorithmus ist ein \emph{Greedy-Algorithmus}
				\item wir probieren immer die g\"unstigste Kante
				\item wir werden sehen, da\ss\ diese Strategie im MST-Problem zum Erfolg f\"uhrt
				\item im allgemeinen sind Greedy-Algorihmen aber \emph{nicht} optimal!
			\end{itemize}
		\end{exampleblock}
	\end{overprint}
\end{frame}

\begin{frame}\frametitle{\mytitle}
	\begin{overprint}
		\onslide<1>
		\begin{block}{Satz}
			Sei $G,c$ ein zusammenh\"angender gewichteter Graph (mit rationalen Gewichten).
			\begin{itemize}
				\item Kruskal hat Laufzeit $O(|E|\log|E|)$
				\item der Algorithmus gibt einen minimal spannenden Baum aus
			\end{itemize}
		\end{block}
		\onslide<2>
		\begin{exampleblock}{Beweis: Laufzeit}
			\begin{itemize}
				\item zur Laufzeit ist nur zu kl\"aren, wie Schritte 5 und 6 implementiert werden
				\item da jeweils die kleinere Komponente der gr\"o\ss eren hinzugef\"ugt wird, verdoppelt sich die Komponentengr\"o\ss e aus Sicht der kleineren Komponente mindestens
				\item jeder Knoten wird also h\"ochstens $O(\log|V|)$ mal einer anderen Komponente zugeordnet
				\item die Laufzeit ist also $O(|V|\log|V|)=O(|E|\log|E|)$
			\end{itemize}
		\end{exampleblock}
		\onslide<3>
		\begin{exampleblock}{Beweis: Korrektheit}
			\begin{itemize}
				\item weil Kruskal nur dann eine Kante hinzuf\"ugt, wenn sie verschiedenene Komponenten von $T$ verbindet, ist die Ausgabe kreisfrei
				\item ferner ist der Ausgabegraph zusammenh\"angend
				\item denn angenommen nicht
				\item dann g\"abe es zwei Komponenten $K_1,K_2$
				\item weil $G$ zusammenh\"angend ist, gibt es eine Kante in $G$, die $K_1,K_2$ verbindet
				\item Kruskal h\"atte also die erste solche Kante einf\"ugen m\"ussen, Widerspruch
			\end{itemize}
		\end{exampleblock}
		\onslide<4>
		\begin{exampleblock}{Beweis: Korrektheit (fortgesetzt)}
			\begin{itemize}
				\item zu zeigen bleibt, da\ss\ die Ausgabe minimales Gewicht hat
				\item angenommen nicht: bezeichne die in $T$ enthaltenen Kanten mit
					\begin{align*}
						e_{i_1},\ldots,e_{i_{n-1}}&&(n=|V|)\\
						i_1\leq\cdots\leq i_{n-1}
					\end{align*}
				\item sei $0\leq k<n-1$ der maximale Index, so da\ss\ ein minimal spannender Baum $$e_{i_1},\ldots,e_{i_k}$$ enth\"alt; sei $B$ so ein minimal spannender Baum
			\end{itemize}
		\end{exampleblock}
		\onslide<5>
		\begin{exampleblock}{Beweis: Korrektheit (fortgesetzt)}
			\begin{itemize}
				\item dann enth\"alt $B$ die Kante $f=e_{i_{k+1}}$ nicht
				\item folglich enth\"alt $(V,E(B)\cup\{f\})$ einen Kreis $C$
				\item dieser enth\"alt eine Kante $f'\in E(B)\setminus E(T)$
				\item $B'=(V,(E(B)\cup\{f\})\setminus\{f'\})$ ist ein spannender Baum
				\item nach Wahl von $k$ gilt $c(B')>c(B)$
				\item daraus folgt, da\ss\ $c(f')<c(f)$
				\item also h\"atte Kruskal $f'$ anstelle von $f$ zu $T$ hinzuf\"ugen m\"ussen
				\item Widerspruch
			\end{itemize}
		\end{exampleblock}
	\end{overprint}
\end{frame}

\begin{frame}\frametitle{\mytitle}
	\begin{overprint}
		\onslide<1>
		\begin{exampleblock}{Der Prim-Algorithmus}
			\begin{itemize}
				\item ein alternativer greedy-Algorithmus zur Berechnung minimaler Spannb\"aume
				\item ausgehend von einem Startknoten $s$ ``z\"uchtet'' der Algorithmus einen spannenden Baum
				\item in jedem Schritt wird der Knoten hinzugef\"ugt, der dem bereits konstruierten Baum mit kleinstem Gewicht hinzugef\"ugt werden kann
				\item effizient implementierbar mit Hilfe von priority queues
			\end{itemize}
		\end{exampleblock}
		\onslide<2>
		\begin{exampleblock}{Algorithmus Prim}
	\begin{enumerate}
		\item F\"ur alle $u\in V(G)$ setze $c(u)=\infty$, $p(u)=\emptyset$.
		\item Setze $c(s)=0$ und $Q=V(G)$.
		\item Solange $Q\neq\emptyset$
		\item $\quad$finde $v\in Q$ mit minimalem $c(v)$ und entferne $v$ aus $Q$
		\item $\quad$f\"ur alle $u\in Q\cap\partial_Gv$
		\item $\quad\quad$falls $c(\{u,v\})<c(u)$, setze $c(u)=c(\{u,v\})$ und $p(u)=v$
		\item Gib die Kantenmenge $\cbc{\{v,p(v)\}:p(v)\neq\emptyset}$ aus.
	\end{enumerate}
		\end{exampleblock}
		\onslide<3>
\begin{block}{Satz}
	Angenommen $G,c$ ist ein zusammenh\"angender gewichteter Graph.
	Dann gibt {\tt Prim}$(G,c)$ die Kantenmenge eines minimalen Spannbaums von $G$ aus.
\end{block}
		\onslide<4>
\begin{exampleblock}{Beweis}
	\begin{itemize}
		\item in jeder Iteration wird ein Knotan aus $Q$ entfernt
		\item daher erfolgen h\"ochstens $|V(G)|$ Iterationen
		\item Schritt (6) stellt sicher, da\ss\ es keine zwei Knoten $x,y\in Q$ mit $p(x)=y$ gibt
		\item daher w\"achst 
	\begin{align*}
		\cE&=\cbc{\{v,p(v)\}:p(v)\neq\emptyset}
	\end{align*}
			in jeder Iteration um eine Kante, mit Ausnahme des ersten Durchlaufs
	\end{itemize}
\end{exampleblock}
		\onslide<5>
\begin{exampleblock}{Beweis}
	\begin{itemize}
		\item \alert{Behauptung:} stets existiert ein minimaler Spannbaum, der $\cE$ enth\"alt
		\item {\em erste Iteration}: $Q=V(G)\setminus\cbc s$ und $\cE=\emptyset$
		\item {\em Induktionsschritt}: sei $T'$ ein MSB, der $\cE'$ aus der vorherigen Iteration enth\"alt
		\item enthalte $\cV'$ die Knoten, die mit $\cE'$ inzident sind
		\item enthalte $\cV=\cV'\cup\cbc v$ die Knoten, die mit $\cE$ inzident sind
		\item sei $e=\{v,p(v)\}$ die zuletzt zu $\cE$ hinzugef\"ugte Kante
		\item angenommen es g\"abe \emph{keinen} MSB, der $\cE$ enth\"alt
	\end{itemize}
\end{exampleblock}
		\onslide<6>
\begin{exampleblock}{Beweis}
	\begin{itemize}
		\item weil $e\not\in E(T')$, enth\"alt $T'+\cbc e$ einen Kreis $C$
		\item auf $C$ gibt es eine Kante $f\neq e$ mit
			\begin{align*}
				f\cap\cV'\neq\emptyset&&\mbox{und}&&f\setminus\cV'\neq\emptyset
			\end{align*}
		\item also gilt $c(f)\geq c(e)$ und $f\not\in\cE$
		\item folglich ist $T=T'-\{f\}+\{e\}$ ein MSB von $G$
		\item Widerspruch
	\end{itemize}
\end{exampleblock}
		\onslide<7>
\begin{exampleblock}{Laufzeit}
	\begin{itemize}
		\item die Menge $Q$ kann mit Hilfe einer min priority queue implementiert werden.
		\item Schritt (2) f\"ugt die Knoten in die min priority queue ein; alle bis auf den Startknoten haben dasselbe Gewicht.
			\item Schritt (4) entfernt jeweils das Minimum aus der queue
			\item Schritt (6) reduziert das Gewicht einiger Nachbarn von $v$
				\item Laufzeit $O(|E(G)|\log|V(G)|)$
	\end{itemize}
\end{exampleblock}
	\end{overprint}
\end{frame}


\begin{frame}\frametitle{\mytitle}
	\begin{overprint}
		\onslide<1>
		\begin{exampleblock}{Anwendung: metrisches TSP}
			\begin{itemize}
\item  wir befassen uns mit einem Spezialfall des TSP-Problems
	\item  sei $G,w$ ein vollst\"andiger gewichteter Graph
	\item wir nehmen an, da\ss\ die Gewichte die {\em Dreiecksungleichung} erf\"ullen, d.h.\
\begin{align*}
	w(\{x,z\})&\leq w(\{x,y\})+w(\{y,z\})&\mbox{f\"ur alle }x,y,z\in V(G).
\end{align*}
\item man spricht vom {\em metrischen TSP}
			\end{itemize}
\end{exampleblock}
		\onslide<2>
		\begin{block}{Proposition}
	Die L\"ange einer optimalen metrischen TSP-Tour ist
	\begin{itemize}
		\item mindestens so hoch wie das Gewicht eines optimalen Spannbaums von $G,w$ und
		\item nicht mehr als doppelt so hoch wie das Gewicht eines optimalen Spannbaums von $G,w$.
	\end{itemize}
\end{block}
	\end{overprint}
\end{frame}

\begin{frame}\frametitle{\mytitle}
	\begin{exampleblock}{Zusammenfassung}
		\begin{itemize}
			\item minimal spannende B\"aume in gewichteten Graphen
			\item Kruskal-Algorithmus
			\item Prim-Algorithmus
			\item Greedy-Strategie
		\end{itemize}
	\end{exampleblock}
\end{frame}

\end{document}
