\PassOptionsToPackage{unicode}{hyperref}
\documentclass[aspectratio=1610, 11pt]{beamer}

\usepackage{amsmath}
\usepackage{amssymb}
\usetheme{tudo}

\title{Datenstrukturen, Algorithmen und Programmierung~2}
\author[A.~Coja-Oghlan]{Amin Coja-Oghlan}
\institute[DAP2]{Lehrstuhl Informatik 2\\Fakult\"at f\"ur Informatik}

\renewcommand{\vec}[1]{\boldsymbol{#1}}
\newcommand\dd{\mathrm d}
\newcommand\eul{\mathrm e}
\newcommand\cA{\mathcal A}
\newcommand\cB{\mathcal B}
\newcommand\cC{\mathcal C}
\newcommand\cD{\mathcal D}
\newcommand\cE{\mathcal E}
\newcommand\cF{\mathcal F}
\newcommand\cG{\mathcal G}
\newcommand\cH{\mathcal H}
\newcommand\cI{\mathcal I}
\newcommand\cJ{\mathcal J}
\newcommand\cK{\mathcal K}
\newcommand\cL{\mathcal L}
\newcommand\cM{\mathcal M}
\newcommand\cN{\mathcal N}
\newcommand\cO{\mathcal O}
\newcommand\cP{\mathcal P}
\newcommand\cQ{\mathcal Q}
\newcommand\cR{\mathcal R}
\newcommand\cS{\mathcal S}
\newcommand\cT{\mathcal T}
\newcommand\cU{\mathcal U}
\newcommand\cV{\mathcal V}
\newcommand\cW{\mathcal W}
\newcommand\cX{\mathcal X}
\newcommand\cY{\mathcal Y}
\newcommand\cZ{\mathcal Z}
\newcommand\fA{\mathfrak A}
\newcommand\fB{\mathfrak B}
\newcommand\fC{\mathfrak C}
\newcommand\fD{\mathfrak D}
\newcommand\fE{\mathfrak E}
\newcommand\fF{\mathfrak F}
\newcommand\fG{\mathfrak G}
\newcommand\fH{\mathfrak H}
\newcommand\fI{\mathfrak I}
\newcommand\fJ{\mathfrak J}
\newcommand\fK{\mathfrak K}
\newcommand\fL{\mathfrak L}
\newcommand\fM{\mathfrak M}
\newcommand\fN{\mathfrak N}
\newcommand\fO{\mathfrak O}
\newcommand\fP{\mathfrak P}
\newcommand\fQ{\mathfrak Q}
\newcommand\fR{\mathfrak R}
\newcommand\fS{\mathfrak S}
\newcommand\fT{\mathfrak T}
\newcommand\fU{\mathfrak U}
\newcommand\fV{\mathfrak V}
\newcommand\fW{\mathfrak W}
\newcommand\fX{\mathfrak X}
\newcommand\fY{\mathfrak Y}
\newcommand\fZ{\mathfrak Z}
\newcommand\fa{\mathfrak a}
\newcommand\fb{\mathfrak b}
\newcommand\fc{\mathfrak c}
\newcommand\fd{\mathfrak d}
\newcommand\fe{\mathfrak e}
\newcommand\ff{\mathfrak f}
\newcommand\fg{\mathfrak g}
\newcommand\fh{\mathfrak h}
%\newcommand\fi{\mathfrak i}
\newcommand\fj{\mathfrak j}
\newcommand\fk{\mathfrak k}
\newcommand\fl{\mathfrak l}
\newcommand\fm{\mathfrak m}
\newcommand\fn{\mathfrak n}
\newcommand\fo{\mathfrak o}
\newcommand\fp{\mathfrak p}
\newcommand\fq{\mathfrak q}
\newcommand\fr{\mathfrak r}
\newcommand\fs{\mathfrak s}
\newcommand\ft{\mathfrak t}
\newcommand\fu{\mathfrak u}
\newcommand\fv{\mathfrak v}
\newcommand\fw{\mathfrak w}
\newcommand\fx{\mathfrak x}
\newcommand\fy{\mathfrak y}
\newcommand\fz{\mathfrak z}
\newcommand\vA{\vec A}
\newcommand\vB{\vec B}
\newcommand\vC{\vec C}
\newcommand\vD{\vec D}
\newcommand\vE{\vec E}
\newcommand\vF{\vec F}
\newcommand\vG{\vec G}
\newcommand\vH{\vec H}
\newcommand\vI{\vec I}
\newcommand\vJ{\vec J}
\newcommand\vK{\vec K}
\newcommand\vL{\vec L}
\newcommand\vM{\vec M}
\newcommand\vN{\vec N}
\newcommand\vO{\vec O}
\newcommand\vP{\vec P}
\newcommand\vQ{\vec Q}
\newcommand\vR{\vec R}
\newcommand\vS{\vec S}
\newcommand\vT{\vec T}
\newcommand\vU{\vec U}
\newcommand\vV{\vec V}
\newcommand\vW{\vec W}
\newcommand\vX{\vec X}
\newcommand\vY{\vec Y}
\newcommand\vZ{\vec Z}
\newcommand\va{\vec a}
\newcommand\vb{\vec b}
\newcommand\vc{\vec c}
\newcommand\vd{\vec d}
\newcommand\ve{\vec e}
\newcommand\vf{\vec f}
\newcommand\vg{\vec g}
\newcommand\vh{\vec h}
\newcommand\vi{\vec i}
\newcommand\vj{\vec j}
\newcommand\vk{\vec k}
\newcommand\vl{\vec l}
\newcommand\vm{\vec m}
\newcommand\vn{\vec n}
\newcommand\vo{\vec o}
\newcommand\vp{\vec p}
\newcommand\vq{\vec q}
\newcommand\vr{\vec r}
\newcommand\vs{\vec s}
\newcommand\vt{\vec t}
\newcommand\vu{\vec u}
%\newcommand\vv{\vec v}
\newcommand\vw{\vec w}
\newcommand\vx{\vec x}
\newcommand\vy{\vec y}
\newcommand\vz{\vec z}
\renewcommand\AA{\mathbb A}
\newcommand\NN{\mathbb N}
\newcommand\ZZ{\mathbb Z}
\newcommand\PP{\mathbb P}
\newcommand\QQ{\mathbb Q}
\newcommand\RR{\mathbb R}
\newcommand\RRpos{\mathbb R_{\geq0}}
\renewcommand\SS{\mathbb S}
\newcommand\CC{\mathbb C}
\newcommand{\ord}{\mathrm{ord}}
\newcommand{\id}{\mathrm{id}}
\newcommand{\pr}{\mathrm{P}}
\newcommand{\Vol}{\mathrm{vol}}
\newcommand\norm[1]{\left\|{#1}\right\|} 
\newcommand\sign{\mathrm{sign}}
\newcommand{\eps}{\varepsilon}
\newcommand{\abs}[1]{\left|#1\right|}
\newcommand\bc[1]{\left({#1}\right)} 
\newcommand\cbc[1]{\left\{{#1}\right\}} 
\newcommand\bcfr[2]{\bc{\frac{#1}{#2}}} 
\newcommand{\bck}[1]{\left\langle{#1}\right\rangle} 
\newcommand\brk[1]{\left\lbrack{#1}\right\rbrack} 
\newcommand\scal[2]{\bck{{#1},{#2}}} 
\newcommand{\vecone}{\mathbb{1}}
\newcommand{\tensor}{\otimes}
\newcommand{\diag}{\mathrm{diag}}
\newcommand{\ggt}{\mathrm{ggT}}
\newcommand{\kgv}{\mathrm{kgV}}
\newcommand{\trans}{\top}
\newcommand{\Karonski}{Karo\'nski}
\newcommand{\Erdos}{Erd\H{o}s}
\newcommand{\Renyi}{R\'enyi}
\newcommand{\Lovasz}{Lov\'asz}
\newcommand{\Juhasz}{Juh\'asz}
\newcommand{\Bollobas}{Bollob\'as}
\newcommand{\Furedi}{F\"uredi}
\newcommand{\Komlos}{Koml\'os}
\newcommand{\Luczak}{\L uczak}
\newcommand{\Kucera}{Ku\v{c}era}
\newcommand{\Szemeredi}{Szemer\'edi}

\begin{document}

\maketitle

\begin{frame}{Arithmetik}
	\begin{exampleblock}{Motivation}
		\begin{itemize}
			\item das Rechnen mit ganzen oder rationalen Zahlen ist ein Grundbaustein vieler Algorithmen
			\item wir stellen die Werkzeuge daf\"ur bereit
		\end{itemize}
	\end{exampleblock}
\end{frame}

\begin{frame}{Arithmetik}
	\begin{exampleblock}{Zahldarstellungen im Rechner}
		\begin{itemize}
			\item Datentypen {\tt int}, {\tt unsigned int}
			\item Flie\ss kommazahlen
			\item rationale Zahlen
		\end{itemize}
	\end{exampleblock}
\end{frame}

\begin{frame}{Arithmetik}
	\begin{exampleblock}{Elementare Rechenoperationen}
		\begin{itemize}
			\item die {\em Darstellungl\"ange} einer Zahl $n\in\NN$ ist $O(\log n)$
			\item Addition/Subtraktion
			\item Division mit Rest
			\item Multiplikation (!)
		\end{itemize}
	\end{exampleblock}
\end{frame}

\begin{frame}{Arithmetik}
	\begin{exampleblock}{Teilbarkeit}
		Seien $x,y\in\ZZ$, $x\neq0$
		\begin{itemize}
			\item $y\mod x$ ist der Divisionsrest
			\item $x$ \emph{teilt} $y$, falls $y\mod x=0$
			\item \emph{Schreibweise:} $x\mid y$
		\end{itemize}
	\end{exampleblock}
\end{frame}

\begin{frame}{Arithmetik}
	\begin{exampleblock}{Gr\"o\ss ter gemeinsamer Teiler}
		Der ggT von $x,y\in\ZZ$, $x\neq0$, ist die gr\"o\ss te Zahl $z\in\NN$, mit $z\mid x$ und $z\mid y$.
	\end{exampleblock}
\end{frame}

\begin{frame}{Arithmetik}
	\begin{exampleblock}{Euklidischer Algorithmus {\tt Euclid}$(a,b)$}
		\begin{enumerate}
			\item falls $a<b$, vertausche $a$ und $b$
			\item setze $a_0=a$, $a_1=b$, $i=1$.
			\item solange $a_i>0$
			\item $\quad$berechne $q_i\in\ZZ$, $a_{i+1}\in\{0,1,\ldots,a_i-1\}$, so da\ss\ $a_{i-1}=q_ia_i+a_{i+1}$.
			\item $\quad$erh\"ohe $i$ um $1$
			\item gib $a_{i-1}$ aus
		\end{enumerate}
	\end{exampleblock}
\end{frame}

\begin{frame}{Arithmetik}
	\begin{block}{Satz}
		{\tt Euclid}$(a,b)$ gibt $\ggt(a,b)$ aus und f\"uhrt $O(\log(|a|+|b|))$ Division aus.
	\end{block}
	\begin{block}{Korollar}
		F\"ur je zwei Zahlen $a,b\in\NN$ gibt es Zahlen $u,v\in\ZZ$, so da\ss\ $\ggt(a,b)=au+bv$.
	\end{block}
\end{frame}

\begin{frame}{Arithmetik}
	\begin{overprint}
		\onslide<1>
		\begin{block}{Definition}
			\begin{itemize}
				\item eine Zahl $z\in\ZZ\setminus\{-1,1\}$ hei\ss t {\em irreduzibel}, falls $y\nmid z$ f\"ur alle $1<y<|z|$.
				\item eine Zahl $z\in\ZZ\setminus\{-1,0,1\}$ hei\ss t {\em Primzahl}, falls f\"ur alle $x,y\in\ZZ$ mit $z|x\cdot y$ gilt, da\ss\ $z|x$ oder $z|y$.
				\item mit $\PP$ wird die Menge aller Primzahlen bezeichnet
			\end{itemize}
		\end{block}
		\onslide<2>
		\begin{block}{Lemma}
			Jede Primzahl ist irreduzibel.
		\end{block}
		\onslide<3>
		\begin{block}{Lemma}
			Jede Zahl $z>1$ besitzt einen irreduziblen Teiler.
		\end{block}
		\begin{block}{Lemma}
			Jede irreduzibele Zahl ist eine Primzahl.
		\end{block}
		\onslide<4>
		\begin{theorem}
			Zu jeder Primzahl $p\in\PP$ gibt es eine Abbildung $w_p:\NN\to\NN_0$, so da\ss\ f\"ur jede nat\"urliche Zahl $z\in\NN$ gilt
			\begin{align*}
				z&=\prod_{p\in\PP}p^{w_p(z)}.
			\end{align*}
			Diese Abbildungen $w_p$ sind eindeutig bestimmt.
		\end{theorem}
	\end{overprint}
\end{frame}

\begin{frame}{Arithmetik}
	\begin{overprint}
		\onslide<1>
\begin{exampleblock}{Modulare Arithmetik}
		Seien $x,y\in\ZZ$ und $m\in\ZZ\setminus\cbc 0$.
		Wir schreiben
		\begin{align*}
			x\equiv y\mod m&&\mbox{falls}&&m\mid x-y
		\end{align*}
		\emph{Sprich:} ``$x$ is kongruent zu $y$ modulo $m$''.
	\end{exampleblock}
\onslide<2>
\begin{block}{Lemma}
	Seien $x,y,x',y'\in\ZZ$ und $m\in\ZZ\setminus\cbc 0$.
	Wenn
	\begin{align*}
		x\equiv y\mod m&&\mbox{und}&&x'\equiv y'\mod m,&&\mbox{dann}\\
		x+x'\equiv y+y'\mod m&&\mbox{und}&&x\cdot x'\equiv y\cdot y'\mod m.
	\end{align*}
\end{block}
\onslide<3>
\begin{block}{Lemma}
	Angenommen $x,y\in\ZZ$, $m,n\in\ZZ\setminus\cbc 0$ und $n\mid m$.	
	Wenn
	\begin{align*}
		x\equiv y\mod m&&\mbox{dann}&&x\equiv y\mod n.
	\end{align*}
\end{block}
\onslide<4>
\begin{block}{Lemma}
	Angenommen $x,y\in\ZZ$, $m,n\in\ZZ\setminus\cbc 0$ und $\ggt(m,n)=1$.	
	Wenn
	\begin{align*}
		x\equiv y\mod m&&\mbox{und}&&x\equiv y\mod n&&\mbox{dann}&&x\equiv y\mod m\cdot n
	\end{align*}
\end{block}
\onslide<5>
\begin{block}{Chinesischer Restsatz}
	Angenommen $m,n\in\NN$ sind relativ prim.
	Dann gibt es f\"ur je zwei ganze Zahlen $x,y$ eine ganze Zahl $z$, so da\ss\
	\begin{align*}
		z\equiv x\mod m&&\mbox{und}&&z\equiv y\mod n.
	\end{align*}
\end{block}
	\end{overprint}
\end{frame}

\begin{frame}{Arithmetik}
	\begin{overprint}
		\onslide<1>
		\begin{exampleblock}{Schnelles Potzenzieren}
			\begin{itemize}
				\item gegeben $x\in\ZZ$ und $\ell,m\in\NN$ suchen wir $z\in\ZZ$ mit
					\begin{align*}
						x^\ell\equiv z\mod m
					\end{align*}
				\item es w\"are offenbar \emph{nicht} effizient, $x^\ell$ durch $\ell$-faches Multiplizieren zu berechnen
			\end{itemize}
		\end{exampleblock}
		\onslide<2>
		\begin{exampleblock}{Algorithmus Schnelles Potenzieren}
			\begin{enumerate}
				\item Bestimme die Darstellung von $\ell$ im Dualsystem: $\ell=\sum_{i=0}^k\ell_i2^i$
				\item Sei $y_0$ der Divisonsrest von $x$ durch $m$.
				\item F\"ur $i=1,\ldots,k$:
				\item $\qquad$sei $y_i$ der Divisionsrest von $y_{i-1}^2$ durch $m$.
				\item Setze $z=1$.
				\item F\"ur $i=0,\ldots,k$:
				\item $\qquad$sei $r$ der Rest von $z\cdot y_i^{\ell_i}$ durch $m$.
				\item $\qquad$setze $z$ auf den Wert $r$.
				\item Gib $z$ aus.
			\end{enumerate}
		\end{exampleblock}
	\end{overprint}
\end{frame}

\begin{frame}{Arithmetik}
	\begin{overprint}
		\onslide<1>
		\begin{exampleblock}{Faktorisieren}
			\begin{itemize}
				\item gegeben $x\in\ZZ$ ist es unser Ziel, die Primfaktorzerlegung von $x$ zu bestimmen
				\item daf\"ur ist derzeit kein effizienter Algorithmus bekannt
				\item wir lernen aber einen Algorithmus kennen, der f\"ur Zahlen $x=pq$ mit $p,q$ prim und $|p-q|$ ``klein'' gut funktioniert
			\end{itemize}
		\end{exampleblock}
		\onslide<2>
		\begin{exampleblock}{Fermat-Faktorisierung}
	{\em Eingabe:} eine ungerade zusammengesetzte Zahl $n>1$.
	\begin{enumerate}
		\item Setze $x=2\lfloor\sqrt n\rfloor+1$, $y=1$, $r=\lfloor\sqrt n\rfloor^2-n$.
		\item Solange $r\neq0$
		\item $\quad$erh\"ohe $r$ um $x$ und anschlie\ss end $x$ um zwei
		\item $\quad$verringere $r$ um $y$ und erh\"ohe anschlie\ss end $y$ um zwei
		\item $\quad$falls $r>0$, gehe zur\"uck zu (4).
		\item gib die Faktorisierung $n=\bcfr{x-y}2\bcfr{x+y-2}2$ aus
	\end{enumerate}
		\end{exampleblock}
	\end{overprint}
\end{frame}

\begin{frame}{Arithmetik}
	\begin{exampleblock}{Zusammenfassung}
		\begin{itemize}
			\item wir haben einige grundlegende Konzepte aus der Zahlentheorie kennengelernt
			\item modulare Arithmetik, euklidischer Algorithmus, chinesischer Restsatz
			\item schnelles Potzenzieren, Fermat-Faktorisierung
		\end{itemize}
	\end{exampleblock}
\end{frame}

 \end{document}
