\PassOptionsToPackage{unicode}{hyperref}
\documentclass[aspectratio=1610, 11pt]{beamer}

\usepackage{amsmath}
\usepackage{amssymb}
\usetheme{tudo}

\title{Datenstrukturen, Algorithmen und Programmierung~2}
\author[A.~Coja-Oghlan]{Amin Coja-Oghlan}
\institute[DAP2]{Lehrstuhl Informatik 2\\Fakult\"at f\"ur Informatik}

\renewcommand{\vec}[1]{\boldsymbol{#1}}
\newcommand\dd{\mathrm d}
\newcommand\eul{\mathrm e}
\newcommand\cA{\mathcal A}
\newcommand\cB{\mathcal B}
\newcommand\cC{\mathcal C}
\newcommand\cD{\mathcal D}
\newcommand\cE{\mathcal E}
\newcommand\cF{\mathcal F}
\newcommand\cG{\mathcal G}
\newcommand\cH{\mathcal H}
\newcommand\cI{\mathcal I}
\newcommand\cJ{\mathcal J}
\newcommand\cK{\mathcal K}
\newcommand\cL{\mathcal L}
\newcommand\cM{\mathcal M}
\newcommand\cN{\mathcal N}
\newcommand\cO{\mathcal O}
\newcommand\cP{\mathcal P}
\newcommand\cQ{\mathcal Q}
\newcommand\cR{\mathcal R}
\newcommand\cS{\mathcal S}
\newcommand\cT{\mathcal T}
\newcommand\cU{\mathcal U}
\newcommand\cV{\mathcal V}
\newcommand\cW{\mathcal W}
\newcommand\cX{\mathcal X}
\newcommand\cY{\mathcal Y}
\newcommand\cZ{\mathcal Z}
\newcommand\fA{\mathfrak A}
\newcommand\fB{\mathfrak B}
\newcommand\fC{\mathfrak C}
\newcommand\fD{\mathfrak D}
\newcommand\fE{\mathfrak E}
\newcommand\fF{\mathfrak F}
\newcommand\fG{\mathfrak G}
\newcommand\fH{\mathfrak H}
\newcommand\fI{\mathfrak I}
\newcommand\fJ{\mathfrak J}
\newcommand\fK{\mathfrak K}
\newcommand\fL{\mathfrak L}
\newcommand\fM{\mathfrak M}
\newcommand\fN{\mathfrak N}
\newcommand\fO{\mathfrak O}
\newcommand\fP{\mathfrak P}
\newcommand\fQ{\mathfrak Q}
\newcommand\fR{\mathfrak R}
\newcommand\fS{\mathfrak S}
\newcommand\fT{\mathfrak T}
\newcommand\fU{\mathfrak U}
\newcommand\fV{\mathfrak V}
\newcommand\fW{\mathfrak W}
\newcommand\fX{\mathfrak X}
\newcommand\fY{\mathfrak Y}
\newcommand\fZ{\mathfrak Z}
\newcommand\fa{\mathfrak a}
\newcommand\fb{\mathfrak b}
\newcommand\fc{\mathfrak c}
\newcommand\fd{\mathfrak d}
\newcommand\fe{\mathfrak e}
\newcommand\ff{\mathfrak f}
\newcommand\fg{\mathfrak g}
\newcommand\fh{\mathfrak h}
%\newcommand\fi{\mathfrak i}
\newcommand\fj{\mathfrak j}
\newcommand\fk{\mathfrak k}
\newcommand\fl{\mathfrak l}
\newcommand\fm{\mathfrak m}
\newcommand\fn{\mathfrak n}
\newcommand\fo{\mathfrak o}
\newcommand\fp{\mathfrak p}
\newcommand\fq{\mathfrak q}
\newcommand\fr{\mathfrak r}
\newcommand\fs{\mathfrak s}
\newcommand\ft{\mathfrak t}
\newcommand\fu{\mathfrak u}
\newcommand\fv{\mathfrak v}
\newcommand\fw{\mathfrak w}
\newcommand\fx{\mathfrak x}
\newcommand\fy{\mathfrak y}
\newcommand\fz{\mathfrak z}
\newcommand\vA{\vec A}
\newcommand\vB{\vec B}
\newcommand\vC{\vec C}
\newcommand\vD{\vec D}
\newcommand\vE{\vec E}
\newcommand\vF{\vec F}
\newcommand\vG{\vec G}
\newcommand\vH{\vec H}
\newcommand\vI{\vec I}
\newcommand\vJ{\vec J}
\newcommand\vK{\vec K}
\newcommand\vL{\vec L}
\newcommand\vM{\vec M}
\newcommand\vN{\vec N}
\newcommand\vO{\vec O}
\newcommand\vP{\vec P}
\newcommand\vQ{\vec Q}
\newcommand\vR{\vec R}
\newcommand\vS{\vec S}
\newcommand\vT{\vec T}
\newcommand\vU{\vec U}
\newcommand\vV{\vec V}
\newcommand\vW{\vec W}
\newcommand\vX{\vec X}
\newcommand\vY{\vec Y}
\newcommand\vZ{\vec Z}
\newcommand\va{\vec a}
\newcommand\vb{\vec b}
\newcommand\vc{\vec c}
\newcommand\vd{\vec d}
\newcommand\ve{\vec e}
\newcommand\vf{\vec f}
\newcommand\vg{\vec g}
\newcommand\vh{\vec h}
\newcommand\vi{\vec i}
\newcommand\vj{\vec j}
\newcommand\vk{\vec k}
\newcommand\vl{\vec l}
\newcommand\vm{\vec m}
\newcommand\vn{\vec n}
\newcommand\vo{\vec o}
\newcommand\vp{\vec p}
\newcommand\vq{\vec q}
\newcommand\vr{\vec r}
\newcommand\vs{\vec s}
\newcommand\vt{\vec t}
\newcommand\vu{\vec u}
%\newcommand\vv{\vec v}
\newcommand\vw{\vec w}
\newcommand\vx{\vec x}
\newcommand\vy{\vec y}
\newcommand\vz{\vec z}
\renewcommand\AA{\mathbb A}
\newcommand\NN{\mathbb N}
\newcommand\ZZ{\mathbb Z}
\newcommand\PP{\mathbb P}
\newcommand\QQ{\mathbb Q}
\newcommand\RR{\mathbb R}
\newcommand\RRpos{\mathbb R_{\geq0}}
\renewcommand\SS{\mathbb S}
\newcommand\CC{\mathbb C}
\newcommand{\ord}{\mathrm{ord}}
\newcommand{\id}{\mathrm{id}}
\newcommand{\pr}{\mathrm{P}}
\newcommand{\Vol}{\mathrm{vol}}
\newcommand\norm[1]{\left\|{#1}\right\|} 
\newcommand\sign{\mathrm{sign}}
\newcommand{\eps}{\varepsilon}
\newcommand{\abs}[1]{\left|#1\right|}
\newcommand\bc[1]{\left({#1}\right)} 
\newcommand\cbc[1]{\left\{{#1}\right\}} 
\newcommand\bcfr[2]{\bc{\frac{#1}{#2}}} 
\newcommand{\bck}[1]{\left\langle{#1}\right\rangle} 
\newcommand\brk[1]{\left\lbrack{#1}\right\rbrack} 
\newcommand\scal[2]{\bck{{#1},{#2}}} 
\newcommand{\vecone}{\mathbb{1}}
\newcommand{\tensor}{\otimes}
\newcommand{\diag}{\mathrm{diag}}
\newcommand{\ggt}{\mathrm{ggT}}
\newcommand{\kgv}{\mathrm{kgV}}
\newcommand{\trans}{\top}
\newcommand{\Karonski}{Karo\'nski}
\newcommand{\Erdos}{Erd\H{o}s}
\newcommand{\Renyi}{R\'enyi}
\newcommand{\Lovasz}{Lov\'asz}
\newcommand{\Juhasz}{Juh\'asz}
\newcommand{\Bollobas}{Bollob\'as}
\newcommand{\Furedi}{F\"uredi}
\newcommand{\Komlos}{Koml\'os}
\newcommand{\Luczak}{\L uczak}
\newcommand{\Kucera}{Ku\v{c}era}
\newcommand{\Szemeredi}{Szemer\'edi}

\begin{document}

\maketitle

\begin{frame}{Rekurrenzen}
	\begin{exampleblock}{Motivation}
		\begin{itemize}
			\item bei der Analyse von Algorithmen begegnen h\"aufig Rekurrenzen
			\item beispielsweise erf\"ullt die Laufzeit von {\tt Quicksort} mit dem {\tt Select}-Algorithmus die Rekurrenz
				\begin{align*}
					\cT_n=O(n)+2\cT_{\lfloor n/2\rfloor}
				\end{align*}
			\item das {\em Master-Theorem} erlaubt es, Rekurrenzen bequem abzusch\"atzen
		\end{itemize}
	\end{exampleblock}
\end{frame}

\begin{frame}{Rekurrenzen}
	\begin{overprint}
		\onslide<1>
	\begin{block}{Master-Theorem}
	Seien $a\geq1$, $b>1$ reelle Zahlen und seien $T:\RRpos\to\RRpos$ und $f:\RRpos\to\RRpos$ Funktionen, so da\ss
	\begin{align*}
		T(x)=aT(x/b)+f(x)\qquad\mbox{f\"ur alle }x\in\RRpos.
	\end{align*}
	\vspace{-5mm}
	\begin{enumerate}[(i)]
		\item falls $f(x)=O(x^{\log_b(a)-\eps})$ f\"ur ein $\eps>0$, dann gilt $T(x)=\Theta(x^{\log_ba})$.
		\item falls $f(x)=\Theta(x^{\log_ba}\log^kx)$ f\"ur ein $k\geq0$, dann gilt $T(x)=\Theta(x^{\log_ba}\log^{k+1}x)$.
		\item falls $f(x)=\Omega(x^{\log_ba+\eps})$ f\"ur ein $\eps>0$ und falls au\ss erdem  Zahlen $c<1$, $x_0>0$ existieren, so da\ss\ $ af(x/b)\leq cf(x)\mbox{ f\"ur alle }x>x_0$, dann gilt $T(x)=\Theta(f(x))$.
	\end{enumerate}
	\end{block}
		\onslide<2>
		\begin{block}{Lemma}
	Seien $a\geq1$ und $b>1$ reelle Zahlen, sei $f:\RRpos\to\RRpos$ eine Funktion und sei $T:\RRpos\to\RRpos$ eine Funktion, so da\ss
	\begin{align}\label{eqlem_master1_1}
		T(x)&=\begin{cases}\Theta(1)&\mbox{falls }0\leq x\leq 1,\\ aT(x/b)+f(x)&\mbox{ sonst.}\end{cases}
	\end{align}
	Dann gilt
	\begin{align}\label{eqlem_master1}
		T(x)&=\Theta(x^{\log_ba})+\sum_{0\leq j\leq\log_bx}a^jf(x/b^j).
	\end{align}
		\end{block}
		\onslide<3>
		\begin{block}{Lemma}
	Seien $a\geq1$ und $b>1$ reelle Zahlen, sei $f:\RRpos\to\RRpos$ eine und sei
	\begin{align}\label{eqlem_master2_1}
		g(x)&=\sum_{0\leq j\leq\log_bx}a^jf(x/b^j)&&(x\geq1).
	\end{align}
	\begin{enumerate}[(i)]
		\item falls $f(x)=O(x^{\log_b(a)-\eps})$ f\"ur ein $\eps>0$, dann gilt $g(x)=\Theta(x^{\log_ba})$.
		\item falls $f(x)=\Theta(x^{\log_ba}\log^kx)$ f\"ur ein $k\geq0$, dann gilt $g(x)=\Theta(x^{\log_ba}\log^{k+1}x)$.
		\item falls $f(x)=\Omega(x^{\log_ba+\eps})$ f\"ur ein $\eps>0$ und falls au\ss erdem  Zahlen $c<1$, $x_0>0$ existieren, so da\ss\ $af(x/b)\leq cf(x)\mbox{ f\"ur alle }x>x_0$, dann gilt $g(x)=\Theta(f(x))$.
	\end{enumerate}
		\end{block}
	\end{overprint}
\end{frame}

\begin{frame}{Rekurrenzen}
	\begin{exampleblock}{Matrixmultiplikation}
		\begin{itemize}
			\item f\"ur zwei $n\times n$-Matrizen $A=(a_{ij})_{i,j=1,\ldots,n}$ and $B=(b_{ij})_{i,j=1,\ldots,n}$ m\"ochten wir das Produkt $C=A\cdot B$ mit Eintr\"agen
				\begin{align*}
					c_{ij}&=\sum_{\ell=1}^na_{i\ell}b_{\ell j}&&(i,j=1,\ldots,n)
				\end{align*}
				berechnen.
			\item der naive Algorithmus ben\"otigt $O(n^3)$ Multiplikationen
		\end{itemize}
	\end{exampleblock}
\end{frame}


\begin{frame}{Rekurrenzen}
	\begin{overprint}
		\onslide<1>
		\begin{exampleblock}{Strassen-Algorithmus}
			\begin{itemize}
				\item wir nehmen an, da\ss\ $n=2^t$ f\"ur $t\in\NN$
				\item {\em divide \&\ conquer:} zerlege $A,B,C$ in Untermatrizen der Gr\"o\ss e $n/2\times n/2$:
					\begin{align*}
						A&=\begin{pmatrix}A_{11}&A_{12}\\A_{21}&A_{22}\end{pmatrix}&
						B&=\begin{pmatrix}B_{11}&B_{12}\\B_{21}&B_{22}\end{pmatrix}&
						C&=\begin{pmatrix}C_{11}&C_{12}\\C_{21}&C_{22}\end{pmatrix}
					\end{align*}
			\end{itemize}
		\end{exampleblock}
		\onslide<2>
		\begin{exampleblock}{Strassen-Algorithmus}
			\begin{itemize}
				\item nun berechne rekursiv die sieben Produkte
					\begin{align*}
						P_1&=(A_{11}+A_{22})(B_{11}+B_{22})\\
						P_2&=(A_{21}+A_{22})B_{11}\\
						P_3&=A_{11}(B_{12}-B_{22})\\
						P_4&=A_{22}(B_{21}-B_{11})\\
						P_5&=(A_{11}+A_{12})B_{22}\\
						P_6&=(A_{21}-A_{11})(B_{11}+B_{12})\\
						P_7&=(A_{12}-A_{22})(B_{21}+B_{22})
					\end{align*}
			\end{itemize}
		\end{exampleblock}
		\onslide<3>
		\begin{exampleblock}{Strassen-Algorithmus}
			\begin{itemize}
				\item dann erhalten wir
					\begin{align*}
						C_{11}&=P_1+P_4-P_5+P_7\\
						C_{12}&=P_3+P_5\\
						C_{21}&=P_2+P_4\\
						C_{22}&=P_1+P_3-P_2+P_6
					\end{align*}
			\end{itemize}
		\end{exampleblock}
		\onslide<4>
		\begin{exampleblock}{Satz}
			Der Strassen-Algorithmus ben\"otigt $T(n)=\Theta(n^{\log_27})$ Rechenoperationen.
		\end{exampleblock}
		\begin{exampleblock}{Beweis}
			\begin{itemize}
				\item die Laufzeit erf\"ullt die Rekurrenz
					\begin{align*}
						T(n)=7T(n/2)+O(n^2).
					\end{align*}
				\item daher folgt die Behauptung aus dem Master-Theorem.
			\end{itemize}
		\end{exampleblock}
	\end{overprint}
\end{frame}

\begin{frame}{Rekurrenzen}
	\begin{exampleblock}{Zusammenfassung}
		\begin{itemize}
			\item das Master-Theorem erm\"oglicht die Absch\"atzung von Funktionen, die durch Rekurrenzen definiert sind
			\item das Theorem gilt entsprechend f\"ur Funktionen $T:\NN\to\RRpos$, wenn in der Rekurrenz $\lfloor\,\cdot\,\rfloor$ oder $\lceil\,\cdot\,\rceil$-Symbole auftreten
			\item {\itshape nicht alle Rekurrenzen lassen sich mit dem Master-Theorem l\"osen}
			\item \emph{Anwendung:} Strassen-Algorithmus
		\end{itemize}
	\end{exampleblock}
\end{frame}

 \end{document}
