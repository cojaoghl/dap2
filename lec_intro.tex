\PassOptionsToPackage{unicode}{hyperref}
\documentclass[aspectratio=1610, 11pt]{beamer}

\usepackage{amsmath}
\usepackage{amssymb}
\usetheme{tudo}

\title{Datenstrukturen, Algorithmen und Programmierung~2}
\author[A.~Coja-Oghlan]{Amin Coja-Oghlan}
\institute[DAP2]{Lehrstuhl Informatik 2\\Fakult\"at f\"ur Informatik}

\newcommand\dd{\mathrm d}
\newcommand\eul{\mathrm e}

\newcommand\cA{\mathcal A}
\newcommand\cB{\mathcal B}
\newcommand\cC{\mathcal C}
\newcommand\cD{\mathcal D}
\newcommand\cE{\mathcal E}
\newcommand\cF{\mathcal F}
\newcommand\cG{\mathcal G}
\newcommand\cH{\mathcal H}
\newcommand\cI{\mathcal I}
\newcommand\cJ{\mathcal J}
\newcommand\cK{\mathcal K}
\newcommand\cL{\mathcal L}
\newcommand\cM{\mathcal M}
\newcommand\cN{\mathcal N}
\newcommand\cO{\mathcal O}
\newcommand\cP{\mathcal P}
\newcommand\cQ{\mathcal Q}
\newcommand\cR{\mathcal R}
\newcommand\cS{\mathcal S}
\newcommand\cT{\mathcal T}
\newcommand\cU{\mathcal U}
\newcommand\cV{\mathcal V}
\newcommand\cW{\mathcal W}
\newcommand\cX{\mathcal X}
\newcommand\cY{\mathcal Y}
\newcommand\cZ{\mathcal Z}

\newcommand\fA{\mathfrak A}
\newcommand\fB{\mathfrak B}
\newcommand\fC{\mathfrak C}
\newcommand\fD{\mathfrak D}
\newcommand\fE{\mathfrak E}
\newcommand\fF{\mathfrak F}
\newcommand\fG{\mathfrak G}
\newcommand\fH{\mathfrak H}
\newcommand\fI{\mathfrak I}
\newcommand\fJ{\mathfrak J}
\newcommand\fK{\mathfrak K}
\newcommand\fL{\mathfrak L}
\newcommand\fM{\mathfrak M}
\newcommand\fN{\mathfrak N}
\newcommand\fO{\mathfrak O}
\newcommand\fP{\mathfrak P}
\newcommand\fQ{\mathfrak Q}
\newcommand\fR{\mathfrak R}
\newcommand\fS{\mathfrak S}
\newcommand\fT{\mathfrak T}
\newcommand\fU{\mathfrak U}
\newcommand\fV{\mathfrak V}
\newcommand\fW{\mathfrak W}
\newcommand\fX{\mathfrak X}
\newcommand\fY{\mathfrak Y}
\newcommand\fZ{\mathfrak Z}

\newcommand\fa{\mathfrak a}
\newcommand\fb{\mathfrak b}
\newcommand\fc{\mathfrak c}
\newcommand\fd{\mathfrak d}
\newcommand\fe{\mathfrak e}
\newcommand\ff{\mathfrak f}
\newcommand\fg{\mathfrak g}
\newcommand\fh{\mathfrak h}
%\newcommand\fi{\mathfrak i}
\newcommand\fj{\mathfrak j}
\newcommand\fk{\mathfrak k}
\newcommand\fl{\mathfrak l}
\newcommand\fm{\mathfrak m}
\newcommand\fn{\mathfrak n}
\newcommand\fo{\mathfrak o}
\newcommand\fp{\mathfrak p}
\newcommand\fq{\mathfrak q}
\newcommand\fr{\mathfrak r}
\newcommand\fs{\mathfrak s}
\newcommand\ft{\mathfrak t}
\newcommand\fu{\mathfrak u}
\newcommand\fv{\mathfrak v}
\newcommand\fw{\mathfrak w}
\newcommand\fx{\mathfrak x}
\newcommand\fy{\mathfrak y}
\newcommand\fz{\mathfrak z}

\newcommand\vA{\vec A}
\newcommand\vB{\vec B}
\newcommand\vC{\vec C}
\newcommand\vD{\vec D}
\newcommand\vE{\vec E}
\newcommand\vF{\vec F}
\newcommand\vG{\vec G}
\newcommand\vH{\vec H}
\newcommand\vI{\vec I}
\newcommand\vJ{\vec J}
\newcommand\vK{\vec K}
\newcommand\vL{\vec L}
\newcommand\vM{\vec M}
\newcommand\vN{\vec N}
\newcommand\vO{\vec O}
\newcommand\vP{\vec P}
\newcommand\vQ{\vec Q}
\newcommand\vR{\vec R}
\newcommand\vS{\vec S}
\newcommand\vT{\vec T}
\newcommand\vU{\vec U}
\newcommand\vV{\vec V}
\newcommand\vW{\vec W}
\newcommand\vX{\vec X}
\newcommand\vY{\vec Y}
\newcommand\vZ{\vec Z}

\newcommand\va{\vec a}
\newcommand\vb{\vec b}
\newcommand\vc{\vec c}
\newcommand\vd{\vec d}
\newcommand\ve{\vec e}
\newcommand\vf{\vec f}
\newcommand\vg{\vec g}
\newcommand\vh{\vec h}
\newcommand\vi{\vec i}
\newcommand\vj{\vec j}
\newcommand\vk{\vec k}
\newcommand\vl{\vec l}
\newcommand\vm{\vec m}
\newcommand\vn{\vec n}
\newcommand\vo{\vec o}
\newcommand\vp{\vec p}
\newcommand\vq{\vec q}
\newcommand\vr{\vec r}
\newcommand\vs{\vec s}
\newcommand\vt{\vec t}
\newcommand\vu{\vec u}
\renewcommand\vv{\vec v}
\newcommand\vw{\vec w}
\newcommand\vx{\vec x}
\newcommand\vy{\vec y}
\newcommand\vz{\vec z}

\renewcommand\AA{\mathbb A}
\newcommand\NN{\mathbb N}
\newcommand\ZZ{\mathbb Z}
\newcommand\PP{\mathbb P}
\newcommand\QQ{\mathbb Q}
\newcommand\RR{\mathbb R}
\newcommand\RRpos{\mathbb R_{\geq0}}
\renewcommand\SS{\mathbb S}
\newcommand\CC{\mathbb C}

\newcommand{\ord}{\mathrm{ord}}
\newcommand{\id}{\mathrm{id}}
\newcommand{\pr}{\mathrm{P}}
\newcommand{\Vol}{\mathrm{vol}}
\newcommand\norm[1]{\left\|{#1}\right\|} 
\newcommand\sign{\mathrm{sign}}
\newcommand{\eps}{\varepsilon}
\newcommand{\abs}[1]{\left|#1\right|}
\newcommand\bc[1]{\left({#1}\right)} 
\newcommand\cbc[1]{\left\{{#1}\right\}} 
\newcommand\bcfr[2]{\bc{\frac{#1}{#2}}} 
\newcommand{\bck}[1]{\left\langle{#1}\right\rangle} 
\newcommand\brk[1]{\left\lbrack{#1}\right\rbrack} 
\newcommand\scal[2]{\bck{{#1},{#2}}} 
\newcommand{\vecone}{\mathbb{1}}
\newcommand{\tensor}{\otimes}
\newcommand{\diag}{\mathrm{diag}}
\newcommand{\ggt}{\mathrm{ggT}}
\newcommand{\kgv}{\mathrm{kgV}}
\newcommand{\trans}{\top}

\newcommand{\Karonski}{Karo\'nski}
\newcommand{\Erdos}{Erd\H{o}s}
\newcommand{\Renyi}{R\'enyi}
\newcommand{\Lovasz}{Lov\'asz}
\newcommand{\Juhasz}{Juh\'asz}
\newcommand{\Bollobas}{Bollob\'as}
\newcommand{\Furedi}{F\"uredi}
\newcommand{\Komlos}{Koml\'os}
\newcommand{\Luczak}{\L uczak}
\newcommand{\Kucera}{Ku\v{c}era}
\newcommand{\Szemeredi}{Szemer\'edi}


\begin{document}

\maketitle

\begin{frame}{Kurze Vorstellung}
	\begin{overprint}
		\onslide<1>
		\begin{exampleblock}{Amin Coja-Oghlan}
		\begin{itemize}
			\item Professor f\"ur effiziente Algorithmen und Komplexit\"atstheorie seit 2021
			\item Lehrstuhl Informatik 2
			\item Vorlesung
		\end{itemize}
	\end{exampleblock}
		\onslide<2>
	\begin{exampleblock}{Arnab Chatterjee, Lena Krieg, und Maurice Rolvien}
		\begin{itemize}
			\item Mitarbeiter am Lehrstuhl Informatik 2
			\item Zust\"andig f\"ur die Organisation der \"Ubungen
		\end{itemize}
	\end{exampleblock}
		\onslide<3>
	\begin{exampleblock}{Erik Thordsen}
		\begin{itemize}
			\item Mitarbeiter am Lehrstuhl Informatik 8
			\item Zust\"andig f\"ur die Organisation des Praktikums (Informatik)
		\end{itemize}
	\end{exampleblock}
		\onslide<4>
	\begin{exampleblock}{Wolfgang Endemann}
		\begin{itemize}
			\item Wissenschaftlicher Mitarbeiter
			\item Fakult\"at für Elektrotechnik und Informationstechnik, Lehrstuhl für Kommunikationstechnik
			\item Zust\"andig f\"ur die Organisation des Praktikums (IKT und ETIT)
		\end{itemize}
	\end{exampleblock}
		\onslide<5>
	\begin{exampleblock}{Ulrike Spear}
		\begin{itemize}
			\item Sekretariat Lehrstuhl Informatik 2
			\item Zust\"andig f\"ur allgemeine organisatorische Fragen und email-Anfragen
		\end{itemize}
	\end{exampleblock}
	\end{overprint}
\end{frame}

\begin{frame}{Vorlesungsseite und Kontakt}
	\begin{exampleblock}{}
		\begin{itemize}
			\item Anfragen richten Sie ausschlie\ss lich per email an
				\begin{quote}
\tt dap2.eac.fk04@tu-dortmund.de
				\end{quote}
			\item Vorlesungshomepage auf
				\begin{quote}
\tt https://ls2-web.cs.tu-dortmund.de/$\sim$mamicoja/dap2.pdf
				\end{quote}
			\item auf der Seite sind Lehrmaterialien (Folien, Skript etc.) zu finden
			\item registerien Sie sich \alert{unbedingt} \"uber LSF!
			\item Sie werden dann \emph{automatisch} in \alert{Moodle} eingeschrieben
			\item wichtige Nachrichten erhalten Sie per email \"uber Moodle
		\end{itemize}
	\end{exampleblock}
\end{frame}


\begin{frame}{Termine}
	\begin{exampleblock}{Klausur}
		\begin{itemize}
			\item 24.7.2022, Zeitraum 15:30--18:45, 90-min.\ Bearbeitungszeit (keine Hilfsmittel)
			\item 28.9.2022, Zeitraum 11:00--14:15, 90-min.\ Bearbeitungszeit (keine Hilfsmittel)
			\item \emph{erfolgreiche Teilnahme an den \"Ubungen ist Zulassungsvoraussetzung!}
			\item wenn Ihnen ein Nachteilsausgleich zusteht, melden Sie uns dies so fr\"uh wie m\"oglich, mindestens aber zwei Wochen vor dem Klausurtermin!
		\end{itemize}
	\end{exampleblock}
\end{frame}

\begin{frame}{Termine}
	\begin{exampleblock}{Vorlesung}
		\begin{itemize}
			\item Di.\ 12:15--13:45, H.001
			\item Do.\ 14:15--15:45, H.001
			\item die Vorlesung findet in Pr\"asenz statt
			\item es gibt keine Vorlesungsaufzeichungen
			\item es gibt keine Videovorlesungen/Webinar
		\end{itemize}
	\end{exampleblock}
\end{frame}

%Uebungsgruppenanmeldung:
%
%Wo: ASSESS https://ess.cs.tu-dortmund.de/ASSESS/
%Wie: Prioritaetenbasiert
%Wann: Jetzt - Donnerstags 23:59
%
%Onlineuebungen(Ausgezeichnet in der Spalte "Name"):
%Mittwochs 8-14,
%Donnerstags 8-10
%
%Ausschliesslich Englische Uebung (Ausgezeichnet in der Spalte "Name"):
%Montags 08-10,
%Donnertsags 8-10
%Mittwochs 8-14,
%Donnerstags 8-10

\begin{frame}{Termine}
	\begin{exampleblock}{\"Ubungen}
		\begin{itemize}
			\item die Einteilung der \"Ubungsgruppen erfolgt \"uber Assess (priorit\"atenbasiert) unter
				\begin{quote}
\tt https://ess.cs.tu-dortmund.de/ASSESS/
				\end{quote}
			\item \emph{Registerierungsdeadline Do 6.~April 23:59.}
			\item die meisten \"Ubungsgruppen finden w\"ochentlich \alert{in Pr\"asenz} statt
			\item es gibt zus\"atzlich mehrere \alert{Online-\"Ubungsgruppen} (Mi 8--14, Do 8--10)
			\item einige \"Ubungsgruppen sind in \alert{Englischer Sprache}; dort sind die L\"osungen auch in Englisch abzugeben (Tutoren sprechen kein Deutsch)!
		\end{itemize}
	\end{exampleblock}
\end{frame}

\begin{frame}{Termine}
	\begin{exampleblock}{\"Ubungen}
		\begin{itemize}
			\item die \"Ubungsaufgaben werden w\"ochtentlich \emph{Montags 8:00} in Moodle eingestellt
			\item die Aufgaben sind bis \emph{Sonntags 23:59} zu bearbeiten und elektronisch \"uber Moodle abzugeben
			\item die Aufgaben werden von den \"Ubungsleitern bewertet
			\item die Bewertungen werden in Moodle gestellt
			\item je Aufgaben k\"onnen 4 Punkte erreicht werden
			\item je \"Ubungsblatt gibt es 4 Aufgaben
			\item Voraussetzung f\"ur eine erfolgreiche Teilnahme ist das Erreichen von \alert{80 \"Ubungspunkten!}
		\end{itemize}
	\end{exampleblock}
\end{frame}

\begin{frame}{Termine}
	\begin{exampleblock}{\"Ubungen}
		\begin{itemize}
			\item Gruppenabgaben sind bis zu 3-er Gruppen \emph{aus der gleichen \"Ubungsgruppe erlaubt}; alle Gruppenmitglieder laden die Abgabe in Moodle hoch!
			\item geben Sie auf jeder Abgabe \emph{Namen und Matrikelnummern} an!
			\item \emph{das erste \"Ubungsblatt kommt am 11.~April}
			\item \emph{die \"Ubungsgruppen starten am 18.~April}
		\end{itemize}
	\end{exampleblock}
\end{frame}

\begin{frame}{Termine}
	\begin{exampleblock}{Praktikum --- Informatik}
		\begin{itemize}
			\item Programmiersprache Java
			\item das Praktikum findet in Pr\"asenz statt in mehreren Gruppen statt
			\item Einteilung \"uber Assess
			\item Organisatoren des Praktikums sind Jonas Ellert und Mart Hagedoorn
			\item erfolgreiches Absolvieren des Praktikums ist Voraussetzung f\"ur den Modulabschlu\ss
			\item \emph{das Informatik-Praktikum ist nicht zul\"assig f\"ur ETIT und IKT-Studis}
		\end{itemize}
	\end{exampleblock}
\end{frame}

\begin{frame}{Termine}
		\hfill\includegraphics[height=30mm]{./images/DAP2_CPP_LSF.png}
	\begin{exampleblock}{Praktikum --- ETIT und IKT}
		\begin{itemize}
			\item Programmiersprache C$++$
			\item Registerierung \"uber LSF mit obigem QR-Code
			\item die Platzvergabe erfolgt am 11.4.2022 ab 14:00 im Raum P1-01-108
			\item ohne Anwesenheit an diesem Tag keine Platzvergabe
			\item \emph{das ETIT und IKT-Praktikum ist nicht zul\"assig f\"ur Informatik-Studis}
		\end{itemize}
	\end{exampleblock}
\end{frame}

\begin{frame}{Inhalte}
	\begin{exampleblock}{Algorithmen und Datenstrukturen}
		\begin{itemize}
			\item ``Programmieren im Kleinen''
			\item Schwerpunkt auf Analyse von Algorithmen in Bezug auf Korrektheit und Resourcenbedarf
			\item mathematische Beweise
			\item Grundlagen aus der Kombinatorik
			\item \emph{die Schwerpunkte und Vorlesungsinhalte sind anders als in den vergangenen Jahren}
		\end{itemize}
	\end{exampleblock}
\end{frame}

\begin{frame}{Inhalte}
	\begin{exampleblock}{Programmier- und IT-F\"ahigkeiten}
		\begin{itemize}
			\item wir werden in der Vorlesung ein wenig ``Unix-Style'' prorgrammieren
			\item (was das genau bedeutet, sehen Sie in den n\"achsten Wochen)
			\item die Unix-Philosophie passt besonders gut mit der Algorithmik zusammen
			\item wir lernen, Algorithmen mit den gew\"unschten Laufzeiten und Speicherbedarfen zu implementieren
			\item au\ss erdem lernen wir einige Tools kennen
			\item dabei wird uns die FOSS-AG der Fachschaft ein wenig unterst\"utzen
		\end{itemize}
	\end{exampleblock}
\end{frame}

\begin{frame}{Kontaktaufnahme}
	\begin{exampleblock}{Fragen?}
		\begin{itemize}
		\item wenden Sie sich jederzeit an {\tt dap2.eac.fk04@tu-dortmund.de}
			\item bei Fragen zu Ihrer konkreten \"Ubungs- oder Praktikumsgruppe, wenden Sie sich per Moodle an Ihren \"Ubungs-/Praktikumsleiter
		\end{itemize}
	\end{exampleblock}
\end{frame}

\end{document}
